\begin{document}
\color{white}
\pagecolor{black_background}

\tableofcontents
\newpage

%youtuber: Johan Montoya
%titulo : Diseño Conceptual en Postensado y realizaciones recientes
%fecha de publicación: 2 noviembre 2020

\section{EL HORMIGON POSTESADO COMO MATERIAL ELASTICO}

El inicio del PT a nivel experimental fue a inicios del siglo
pasado para evitar la fisuración. \newline

Los primeros experimentos fueron fallidos por las pérdidas diferidas
y por los materiales de baja resistencia en la época. \newline

Eugène Freyssinet fue el primer ingeniero en resolver con éxitos prácticos
y patentado su sistema en el año 1928. \newline

En el calculo de un elemento de hormigon armado se debe realizar
tomando en cuenta la inercia fisurada de la sección. Mientras que un
elemento de hormigon postesado este puede ser calculado con la inercia
bruta. \newline

Por ejemplo en una losa (de hormigon armado) de 2m de voladizo podemos tener en el calculo
una flecha de 2mm y luego en obra nos da una flecha de 23mm, lo que utilizamos en
nuestros calculos la inercia de la sección bruta.


\section{COMPORTAMIENTO INSTANTANEO. EL PT COMO UNA ACCIÓN}

Pretensar una estructura es aplicar un sistema de cargas artificialmente
creadas que contrarresten el efecto del peso propio y sobrecargas
actuantes.

\begin{itemize}
    \item Carga centrada recta
    \item Carga excéntrica recta
        \subitem - Dentro del nucleo central (compresión compuesta)
        \subitem - Fuera del nucleo cental (flexión compuesta)
    \item Trazado curvo
		\subitem {- Se puede adaptar exactamente al diagrama de momentos flectores
				  y evitar esfuerzo de tracción en toda la pieza.}
\end{itemize}

PERDIDAS INSTANTANEAS:

\begin{itemize}
	\item Fricción entre cables y vainas.
	\item Deformación elástica del concreto.
	\item Penetración de cuñas.
	\item Fricción en los anclajes.
\end{itemize}

\begin{figure}[H]
\centering
\import{img/}{1.pdf_tex}
\end{figure}

Podemos mover el cable en una sección rectangular de viga por ejemplo
para que esta a h/3 desde la base inferior ya que ese es el limite
para que no nos genere tracción en la fibra superior, es decir
tomando en cuenta el concepto de nucleo central de la sección.

\begin{figure}[H]
\centering
\import{img/}{2.pdf_tex}
\end{figure}




%\section{COMPORTAMIENTO EN EL TIEMPO. DEFORMACIONES IMPUESTAS}

%\section{EL CONCEPTO DE CARGA BALANCEADA}

%\section{FUERZA CONCENTRADA EN LOS ANCLAJES}

%\section{EL POSTESADO Y LOS MATERIALES DE ALTA RESISTENCIA}

%\section{CARGAS EN VACIO. LA IMPORTANCIA DEL PROCESO CONSTRUCTIVO}

%\section{LIMITACIONES DEL POSTESADO EN ZONAS SISMICAS}


\end{document}
